\documentclass[10pt,a4paper]{article}
\usepackage[utf8]{inputenc}
\usepackage{amsmath}
\usepackage{amsfonts}
\usepackage{amssymb}
\usepackage{color}
\usepackage{listings}
\usepackage[hmargin=3cm,vmargin=3.5cm]{geometry}
\definecolor{Brown}{cmyk}{0,0.81,1,0.60}
\definecolor{OliveGreen}{cmyk}{0.64,0,0.95,0.40}
\definecolor{CadetBlue}{cmyk}{0.62,0.57,0.23,0}

\author{Samuel Rhody, Kevin O'Connor, and Benjamin Kaiser}
\title{Cryptography \& Network Security I Project \ SKB Protocol}

\begin{document}

\lstset{ %
language=C++,                % choose the language of the code
basicstyle=\footnotesize,       % the size of the fonts that are used for the code
backgroundcolor=\color{white},  % choose the background color.
showspaces=false,               % show spaces adding particular underscores
showstringspaces=false,         % underline spaces within strings
showtabs=false,                 % show tabs within strings adding particular underscores
tabsize=2,          % sets default tabsize to 2 spaces
captionpos=b,           % sets the caption-position to bottom
breaklines=true,        % sets automatic line breaking
breakatwhitespace=false,    % sets if automatic breaks should only happen at whitespace
escapeinside={\%*}{*)}          % if you want to add a comment within your code
}

\maketitle{}

\section{Library}

The library we used for our implementation is Crypto++. Documentation is available at\\
http://www.cryptopp.com/docs/ref/. For a list of the specific files we included, see util.cpp:13-18 or below:\\

\begin{lstlisting}
#include "includes/cryptopp/sha.h"
#include "includes/cryptopp/hex.h"
#include "includes/cryptopp/aes.h"
#include "includes/cryptopp/ccm.h"
#include "includes/cryptopp/gcm.h"
#include "includes/cryptopp/osrng.h"
\end{lstlisting}

\section{Compiling from Source and Launching}

The makefile contains all necessary commands to compile from source, although compilation will require a working install of Crypto++. Executables for the Bank, ATM, and Proxy have all been provided as well. \\

To launch the interface, first run the Bank and Proxy files. Bank takes a single argument: a port number to connect to the proxy. Proxy takes two arguments: the port the ATM will connect to and the port the Bank will connect to. ATM takes two arguments: the port it will connect to the Proxy through and an ATM number, 1 through 50. These numbers represent individual physical ATMs. If you would like to connect multiple ATMs, you must use different ATM numbers.\\

A usage example follows (execute each command in a separate terminal window):\\

\begin{lstlisting}
./bank 4444
./proxy 6666 4444
./atm 6666 1
\end{lstlisting}

\section{Executing Transactions}

The ATM is equipped with the following transactions: login, logout, balance, withdraw, transfer. The Bank is equipped with the following transactions: balance, deposit.

\subsection{ATM Commands}

The login command accepts one argument: the username of the account to be logged in to. It searches for a matching .card file, then prompts the user for the PIN, which is six digits. The PIN is masked with asterixes during input. The accounts and PINs are as follows:\\

\begin{lstlisting}
Username		  PIN		Initial Balance
	alice			123456		$100.00
	bob				234567		$50.00
	eve				345678		$0.00
\end{lstlisting}

If the login fails for any reason (invalid user or invalid PIN), the user will still be allowed to proceed, but at the end of their transaction will get a "transaction denied" error. This is to avoid leaking any information about the accounts.\\

If a login is attempted with a valid user but invalid PIN three times on the same user, the user's account will be locked by the bank. This can only be cleared by resetting the bank server - something we assume realistic adversaries would not be able to do.\\

The logout command terminates the ATM process immediately.\\

The balance command takes no arguments and returns the current balance of the logged-in user. The user is then logged out.\\

The withdraw command takes one argument: the amount to be withdrawn. If the argument is a valid amount, the account balance is deducted by that amount and the new balance is printed. If the argument is invalid, the user gets a "transaction denied" error. The user is logged out at the end of this command.\\

The transfer command takes two arguments: the account to be transferred to and the amount to be transfered. If either argument is invalid (arg1 is not a valid account or arg2 is an invalid amount), the user gets a "transaction denied" error. The user is logged out at the end of this command.

\subsection{Bank Commands}

The balance command takes one argument: the username of the account in question. It returns the current balance of that user.\\

The deposit command takes two arguments: the username of the account to deposit to and the amount to deposit. If both arguments are valid, the amount is deposited into the account.\\

\section{Communications \& Packet Handling}

Communications between the ATM and Bank are shown in the table below. Nonces are referred to as $N_b$ for the Bank and $N_a$ for the ATM. The message is referred to as $M$. Login information is referred to as $L$. Hashing is referred to as $H()$. Return messages are referred to as $r$.\\

\begin{tabular}{|c|ccc|}
	\hline
Step &  ATM  & Direction & Bank \\
	\hline
1   & $getNonce()$ & $\rightarrow$ & Generate $N_b$ \& create session  \\\\
2   & Store $N_b$ & $\leftarrow$ & $E(N_b)$  \\\\
3   & $E_x(L|N_a|H(N_a + N_b))$ & $\rightarrow$ & $D_x(M)$, $validateNonce()$,\\
	& & & $validateLogin(L)$ \\\\
4   & $D_x(L|N_b)$, $validateNonce()$ & $\leftarrow$ & $Generate N_b$, $E_x(M|N_b|H(N_a+N_b))$ \\  & == "I Got That" & &\\\\
5   & Generate $N_a$, & $\rightarrow$ & $D_x(M)$, $validateNonce()$, \\
	& command = $E_x(M|N_a|H(N_a+N_b))$ & & $executeCommand()$, return r\\\\
6   & $D_x(L|N_b)$ and print r & $\leftarrow$ & Generate $N_b$, \\
	& & & $E_x(r|N_b|H(N_a+N_b))$, kill session\\
	\hline
\end{tabular}

{\footnotesize Table 3.1: Communications between ATM and Bank}\\

During this whole transaction process, both the Bank and the ATM are keeping track of their state. The states are as follows:\\

\begin{tabular}{|c|c|}
	\hline
Num & State\\
	\hline
0 & ATM requests. initial nonce from the bank and the two handshake.\\
1 & Bank sends back a nonce\\
2 & ATM sends login information\\
3 & Bank sends an ACK\\
4 & ATM sends a valid command for the bank\\
5 & Bank replies with the response to command or the generic error message\\
	\hline
\end{tabular}

{\footnotesize Table 3.2: List of states that exist during transactions}\\

After the fifth state, both the ATM and the Bank go back to state 0. At any point, the Bank can send a kill message to the ATM that will terminate the transaction and send both back to state 0.

\section{Encryption, Padding, and Delays}

Before being sent, all packets are padded to length 1022, encrypted by AES, then delayed by a random amount of time (less than 1 second) before being sent.  

\end{document}